\documentclass[12pt]{article}
\title{Project 1}
\author{Scott Reyes, Christian Suasi}

\begin{document}
	\maketitle
	\section*{a.}
	\section*{b.}
	\section*{c.}
	\section*{d. Heuristics}
	The heuristic that I chose was the Euclidean distance between the node being considered and the end node. For any two points,there is no shorter distance than: $$D=\sqrt[2]{x^2+y^2}$$
	Thus using this heuristic, we can never overestimate the distance between the two nodes. This heuristic is consistent(and consequently admissible) because the graph is organized by integer coordinates similar to a Euclidean plane. Calculating the heuristic between them would have to be quick which - while the formula is short - the square root function is not.

	Other possibilities for heuristics would include:

	Manhattan distance:	the cost of traversal that can be made using the difference of the x coordinate of the current node and the goal node, and the difference of the y coordinate of the current node and the goal node
	\section*{e.}
	\section*{f. Cost}
	Uniform Cost search will essentially be on par with Dijkstra's search algorithm. A* will perform generally better due to assigning higher heuristic costs to the other nodes, thus discouraging the expansion of nodes.
\end{document}
